\documentclass[10pt]{article}

\usepackage[margin=1in]{geometry}
\usepackage{amsmath}
\usepackage{amssymb}
\usepackage{amsthm}
\usepackage{mathtools}
\usepackage[shortlabels]{enumitem}
\usepackage[normalem]{ulem}
\usepackage{courier}

\usepackage{hyperref}
\hypersetup{
  colorlinks   = true, %Colours links instead of ugly boxes
  urlcolor     = black, %Colour for external hyperlinks
  linkcolor    = blue, %Colour of internal links
  citecolor    = blue  %Colour of citations
}

\usepackage[T1]{fontenc}
\usepackage{listings}
\lstset{
    language=HTML
    ,basicstyle=\linespread{1}\ttfamily
    ,keywordstyle=
    %,numbers=left
    ,breaklines=true
    }

%%%%%%%%%%%%%%%%%%%%%%%%%%%%%%%%%%%%%%%%%%%%%%%%%%%%%%%%%%%%%%%%%%%%%%%%%%%%%%%%

\theoremstyle{definition}
\newtheorem{problem}{Problem}
\newtheorem{note}{Note}
\newcommand{\E}{\mathbb E}
\newcommand{\R}{\mathbb R}
\DeclareMathOperator{\Var}{Var}
\DeclareMathOperator*{\argmin}{arg\,min}
\DeclareMathOperator*{\argmax}{arg\,max}

\newcommand{\trans}[1]{{#1}^{T}}
\newcommand{\loss}{\ell}
\newcommand{\w}{\mathbf w}
\newcommand{\mle}[1]{\hat{#1}_{\textit{mle}}}
\newcommand{\map}[1]{\hat{#1}_{\textit{map}}}
\newcommand{\normal}{\mathcal{N}}
\newcommand{\x}{\mathbf x}
\newcommand{\y}{\mathbf y}
\newcommand{\ltwo}[1]{\lVert {#1} \rVert}

%%%%%%%%%%%%%%%%%%%%%%%%%%%%%%%%%%%%%%%%%%%%%%%%%%%%%%%%%%%%%%%%%%%%%%%%%%%%%%%%

\begin{document}
\begin{center}
    {
\Large
    Quiz: Functions (Practice Problems)
}

    \vspace{0.1in}
\end{center}

\vspace{0.15in}
\noindent
%\textbf{Note:}
\begin{note}
The format of this quiz will be the same as the previous quiz (4 problems, each worth 1 point).
1 of these problems will be taken from the material from topic 0,
and the other 3 problems will be on functions following the problems below.
\end{note}
\vspace{0.15in}

\section{Functional Programming}
\begin{note}
    I will generate new problems from the problems below by: (1) changing the expressions in the lambda functions, (2) changing the values in the range function.
\end{note}
\vspace{0.15in}

\begin{problem}
    Write the output of the final command in the following terminal session.
    If the command has no output, then leave the problem blank.
\end{problem}
\begin{lstlisting}
$ cd
$ rm -rf quiz
$ mkdir quiz
$ cd quiz
$ cat > foo.py <<EOF
def foo(x):
    return x+1
xs = [1, 2, 3]
xs = [foo(x) for x in xs]
print("xs=", xs)
EOF
$ python3 foo.py
\end{lstlisting}
\vspace{0.4in}

\begin{problem}
    Write the output of the final command in the following terminal session.
    If the command has no output, then leave the problem blank.
\end{problem}
\begin{lstlisting}
$ cd
$ rm -rf quiz
$ mkdir quiz
$ cd quiz
$ cat > foo.py <<EOF
def foo(x):
    return x+1
xs = [1, 2, 3]
xs = map(foo, xs)
xs = list(xs)
print("xs=", xs)
EOF
$ python3 foo.py
\end{lstlisting}
\vspace{0.4in}

\newpage
\begin{problem}
    Write the output of the final command in the following terminal session.
    If the command has no output, then leave the problem blank.
\end{problem}
\begin{lstlisting}
$ cd
$ rm -rf quiz
$ mkdir quiz
$ cd quiz
$ cat > foo.py <<EOF
xs = range(3, 5)
xs = map(lambda x: x+1, xs)
xs = list(xs)
print("xs=", xs)
EOF
$ python3 foo.py
\end{lstlisting}
\vspace{0.4in}

\begin{problem}
    Write the output of the final command in the following terminal session.
    If the command has no output, then leave the problem blank.
\end{problem}
\begin{lstlisting}
$ cd
$ rm -rf quiz
$ mkdir quiz
$ cd quiz
$ cat > foo.py <<EOF
foo = lambda x: x*2
xs = range(5, 1, -2)
xs = map(foo, xs)
xs = list(xs)
print("xs=", xs)
EOF
$ python3 foo.py
\end{lstlisting}
\vspace{0.4in}

\begin{problem}
    Write the output of the final command in the following terminal session.
    If the command has no output, then leave the problem blank.
\end{problem}
\begin{lstlisting}
$ cd
$ rm -rf quiz
$ mkdir quiz
$ cd quiz
$ cat > foo.py <<EOF
foo = lambda x: x < 5
xs = range(10)
xs = [x for x in xs if foo(x)]
xs = list(xs)
print("xs=", xs)
EOF
$ python3 foo.py
\end{lstlisting}
\vspace{0.4in}

\begin{problem}
    Write the output of the final command in the following terminal session.
    If the command has no output, then leave the problem blank.
\end{problem}
\begin{lstlisting}
$ cd
$ rm -rf quiz
$ mkdir quiz
$ cd quiz
$ cat > foo.py <<EOF
foo = lambda x: x < 5
xs = range(10)
xs = filter(foo, xs)
xs = list(xs)
print("xs=", xs)
EOF
$ python3 foo.py
\end{lstlisting}
\vspace{0.4in}

\begin{problem}
    Write the output of the final command in the following terminal session.
    If the command has no output, then leave the problem blank.
\end{problem}
\begin{lstlisting}
$ cd
$ rm -rf quiz
$ mkdir quiz
$ cd quiz
$ cat > foo.py <<EOF
xs = range(10)
xs = [x*2 for x in xs if x<5]
xs = list(xs)
print("xs=", xs)
EOF
$ python3 foo.py
\end{lstlisting}
\vspace{0.4in}

\begin{problem}
    Write the output of the final command in the following terminal session.
    If the command has no output, then leave the problem blank.
\end{problem}
\begin{lstlisting}
$ cd
$ rm -rf quiz
$ mkdir quiz
$ cd quiz
$ cat > foo.py <<EOF
xs = range(10)
xs = map(lambda x: x*2, xs)
xs = filter(lambda x: x<5, xs)
xs = list(xs)
print("xs=", xs)
EOF
$ python3 foo.py
\end{lstlisting}
\vspace{0.4in}

\begin{problem}
    Write the output of the final command in the following terminal session.
    If the command has no output, then leave the problem blank.
\end{problem}
\begin{lstlisting}
$ cd
$ rm -rf quiz
$ mkdir quiz
$ cd quiz
$ cat > foo.py <<EOF
xs = range(10)
xs = filter(lambda x: x<5, xs)
xs = map(lambda x: x*2, xs)
xs = list(xs)
print("xs=", xs)
EOF
$ python3 foo.py
\end{lstlisting}
\vspace{0.4in}

\begin{problem}
    Write the output of the final command in the following terminal session.
    If the command has no output, then leave the problem blank.
\end{problem}
\begin{lstlisting}
$ cd
$ rm -rf quiz
$ mkdir quiz
$ cd quiz
$ cat > foo.py <<EOF
xs = map(lambda x: x*2, filter(lambda x: x<5, range(10)))
xs = list(xs)
print("xs=", xs)
EOF
$ python3 foo.py
\end{lstlisting}
\vspace{0.4in}

\begin{problem}
    Write the output of the final command in the following terminal session.
    If the command has no output, then leave the problem blank.
\end{problem}
\begin{lstlisting}
$ cd
$ rm -rf quiz
$ mkdir quiz
$ cd quiz
$ cat > foo.py <<EOF
xs = filter(lambda x: x<5, map(lambda x: x*2, range(10)))
xs = list(xs)
print("xs=", xs)
EOF
$ python3 foo.py
\end{lstlisting}
\vspace{0.4in}

\newpage
\section{Scope}
\begin{note}
    I will generate new problems from the problems below by: (1) adding/removing the global keyword, (2) changing constants in the python code, or (3) re-ordering or duplicating lines.
\end{note}

\begin{problem}
    Write the output of the final command in the following terminal session.
    If the command has no output, then leave the problem blank.
\end{problem}
\begin{lstlisting}
$ cd
$ rm -rf quiz
$ mkdir quiz
$ cd quiz
$ cat > foo.py <<EOF
x = 1
def foo(x):
    return x + 1
x = 2
print('foo(3)=', foo(3))
EOF
$ python3 foo.py
\end{lstlisting}
%\vspace{0.4in}


\begin{problem}
    Write the output of the final command in the following terminal session.
    If the command has no output, then leave the problem blank.
\end{problem}
\begin{lstlisting}
$ cd
$ rm -rf quiz
$ mkdir quiz
$ cd quiz
$ cat > foo.py <<EOF
x = 1
def foo(y):
    x = 3
    return y + x
print('foo(3)=', foo(3))
EOF
$ python3 foo.py
\end{lstlisting}
%\vspace{0.4in}

\begin{problem}
    Write the output of the final command in the following terminal session.
    If the command has no output, then leave the problem blank.
\end{problem}
\begin{lstlisting}
$ cd
$ rm -rf quiz
$ mkdir quiz
$ cd quiz
$ cat > foo.py <<EOF
x = 1
def foo(y):
    x = 3
    return y + x
foo(5)
print('x=', x)
EOF
$ python3 foo.py
\end{lstlisting}
%\vspace{0.4in}

\begin{problem}
    Write the output of the final command in the following terminal session.
    If the command has no output, then leave the problem blank.
\end{problem}
\begin{lstlisting}
$ cd
$ rm -rf quiz
$ mkdir quiz
$ cd quiz
$ cat > foo.py <<EOF
x = 1
def foo(x):
    return x + 1
foo(5)
print('x=', x)
EOF
$ python3 foo.py
\end{lstlisting}
\vspace{0.4in}


\begin{problem}
    Write the output of the final command in the following terminal session.
    If the command has no output, then leave the problem blank.
\end{problem}
\begin{lstlisting}
$ cd
$ rm -rf quiz
$ mkdir quiz
$ cd quiz
$ cat > foo.py <<EOF
x = 1
def foo(y):
    global x
    x = y
foo(5)
print('x=', x)
EOF
$ python3 foo.py
\end{lstlisting}
\vspace{0.4in}


\begin{problem}
    Write the output of the final command in the following terminal session.
    If the command has no output, then leave the problem blank.
\end{problem}
\begin{lstlisting}
$ cd
$ rm -rf quiz
$ mkdir quiz
$ cd quiz
$ cat > foo.py <<EOF
xs = [1, 2, 3]
def foo(y):
    xs.append(y)
foo(5)
print('sum(xs)=',sum(xs))
EOF
$ python3 foo.py
\end{lstlisting}
\vspace{0.4in}


\begin{problem}
    Write the output of the final command in the following terminal session.
    If the command has no output, then leave the problem blank.
\end{problem}
\begin{lstlisting}
$ cd
$ rm -rf quiz
$ mkdir quiz
$ cd quiz
$ cat > foo.py <<EOF
xs = [1, 2, 3]
def foo(y):
    xs = [y, y, y]
    return y
foo(5)
print('sum(xs)=',sum(xs))
EOF
$ python3 foo.py
\end{lstlisting}
\vspace{0.4in}


\begin{problem}
    Write the output of the final command in the following terminal session.
    If the command has no output, then leave the problem blank.
\end{problem}
\begin{lstlisting}
$ cd
$ rm -rf quiz
$ mkdir quiz
$ cd quiz
$ cat > foo.py <<EOF
xs = [1, 2, 3]
def foo(y):
    global xs
    xs = [y, y, y]
    return y
foo(5)
print('sum(xs)=',sum(xs))
EOF
$ python3 foo.py
\end{lstlisting}
\vspace{0.4in}


\newpage
\begin{problem}
    Write the output of the final command in the following terminal session.
    If the command has no output, then leave the problem blank.
\end{problem}
\begin{lstlisting}
$ cd
$ rm -rf quiz
$ mkdir quiz
$ cd quiz
$ cat > foo.py <<EOF
xs = [1, 2, 3]
def foo():
    xs.pop()
def bar():
    global xs
    xs = [4, 5, 6]
    xs.append(7)
foo()
bar()
foo()
bar()
foo()
foo()
print('sum(xs)=',sum(xs))
EOF
$ python3 foo.py
\end{lstlisting}
\vspace{0.4in}


\begin{problem}
    Write the output of the final command in the following terminal session.
    If the command has no output, then leave the problem blank.
\end{problem}
\begin{lstlisting}
$ cd
$ rm -rf quiz
$ mkdir quiz
$ cd quiz
$ cat > foo.py <<EOF
xs = [1, 2, 3]
def foo():
    global xs
    xs.pop()
def bar():
    global xs
    xs = [4, 5, 6]
    xs.append(7)
foo()
bar()
foo()
bar()
foo()
foo()
print('sum(xs)=',sum(xs))
EOF
$ python3 foo.py
\end{lstlisting}
\vspace{0.4in}


%\begin{problem}
    %Write the output of the final command in the following terminal session.
    %If the command has no output, then leave the problem blank.
%\end{problem}
%\begin{lstlisting}
%$ cd
%$ rm -rf quiz
%$ mkdir quiz
%$ cd quiz
%$ cat > foo.py <<EOF
%xs = [1, 2, 3]
%def foo():
    %global xs
    %xs.pop()
%def bar():
    %xs = [4, 5, 6]
    %xs.append(7)
%foo()
%bar()
%foo()
%bar()
%foo()
%foo()
%print('sum(xs)=',sum(xs))
%EOF
%$ python3 foo.py
%\end{lstlisting}
%\vspace{0.4in}

\end{document}
